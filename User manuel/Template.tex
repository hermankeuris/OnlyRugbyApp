\documentclass[hidelinks,a4paper,12pt]{article}
\addtolength{\oddsidemargin}{-1.cm}
\addtolength{\textwidth}{2cm}
\addtolength{\topmargin}{-2cm}
\addtolength{\textheight}{3.5cm}
\newcommand{\HRule}{\rule{\linewidth}{0.5mm}}
\makeindex

\usepackage{longtable}
\usepackage[pdftex]{graphicx}
\usepackage{makeidx}
\usepackage{hyperref}
\hypersetup{
    colorlinks=true,
    linkcolor=blue,
    filecolor=magenta,      
    urlcolor=cyan,
}


% define the title
\author{Men-at-Work}
\title{ OnlyRugby User Manuel: Demo 2}
\begin{document}
\setlength{\parskip}{6pt}

% generates the title
\begin{titlepage}

\begin{center}
% Upper part of the page       
\includegraphics[width=1\textwidth]{./up-logo.jpg}\\[0.4cm]    
\textsc{\LARGE Department of Computer Science}\\[1.5cm]
\textsc{\Large COS 301 - Software Engineering}\\[0.5cm]
% Title
\HRule \\[0.4cm]
{ \huge \bfseries OnlyRugby\\User Manuel: Demo 2}\\[0.4cm]
\HRule \\[0.4cm]
% Author and supervisor
\begin{minipage}{0.4\textwidth}
\begin{flushleft} \large
\emph{Authors:}
\end{flushleft}
\end{minipage}
\begin{minipage}{0.4\textwidth}
\begin{flushright} \large
\emph{Student number:}
\end{flushright}
\end{minipage}

\begin{minipage}{0.4\textwidth}
\begin{flushleft} \large
Herman {Keuris}
\end{flushleft}
\end{minipage}
\begin{minipage}{0.4\textwidth}
\begin{flushright} \large
\emph{}
u13037618
\end{flushright}
\end{minipage}

\begin{minipage}{0.4\textwidth}
\begin{flushleft} \large
Johan {van Rooyen}
\end{flushleft}
\end{minipage}
\begin{minipage}{0.4\textwidth}
\begin{flushright} \large
\emph{}
u11205131
\end{flushright}
\end{minipage}

\begin{minipage}{0.4\textwidth}
\begin{flushleft} \large
Estian {Rosslee}
\end{flushleft}
\end{minipage}
\begin{minipage}{0.4\textwidth}
\begin{flushright} \large
\emph{}
u12223426
\end{flushright}
\end{minipage}

\begin{minipage}{0.4\textwidth}
\begin{flushleft} \large
Ivan {Henning}
\end{flushleft}
\end{minipage}
\begin{minipage}{0.4\textwidth}
\begin{flushright} \large
\emph{}
u13008219
\end{flushright}
\end{minipage}

\begin{minipage}{0.4\textwidth}
\begin{flushleft} \large
Muller {Potgieter}
\end{flushleft}
\end{minipage}
\begin{minipage}{0.4\textwidth}
\begin{flushright} \large
\emph{}
u12003672
\end{flushright}
\end{minipage}

\vfill
% Bottom of the page
{\large \today}
\end{center}
\end{titlepage}
\footnotesize
%\input{declaration_of_originality.tex}
\normalsize


\pagenumbering{roman}
\tableofcontents
\newpage
\pagenumbering{arabic}

\newpage
\section{User Manuel}
\subsection{System Overview} 
The OnlyRugby App will serve school rugby teams and rugby clubs who have registered their teams with the main OnlyRugby website (implemented by our clients) by allowing them to log basic match time information. This means that when the user's team is playing a match the user can log such information as which team and players have scored, any injuries or penalties during the game and which team has won the line-out/scrum etc. This information is then stored on the main website's database allowing user's to compare statistics through the site.
\subsection{System Configuration}
\begin {itemize}
	\item The user will run the app itself from a phone using Android 4.0 or higher.
	\item All communication between the mobile app and the server will occur through HTTP and will consist of JSON objects which will be processed by a RESTful API on the server side (which will then format the information to be stored on the main OnlyRugby website's database) and the main app on the user's mobile device respectivley.
	\item The mobile app will also manage a small temporary SQLite database which will store any information waiting to be sent to the server when a connection is again established.
	\item The server will also use SMTP to send emails to an email account specified by the user in certain situations (e.g. when the user first registers an account with OnlyRugby). 
\end{itemize}
\begin{center}
  	 \includegraphics[width=0.5\textwidth] {./Sysyem Configuration.jpg}\\[0.4cm]
\end{center}
\subsection{Installation}
\begin {itemize}
	\item The user will have to choose to download the app from the Google App store and the rest will be done automatically by the app store itself.
	\item The user will be notified when the app finished downloading and is ready to run.
\end{itemize}

\subsection{Getting started}
	
\subsection{Using the system}
The functionality of the app will be spread between the following use cases:
	\subsection{Login/Logout}
		\begin {itemize}
			\item The first screen after running the App the user will be presented with two edit fields one for his username and one for his password, the user enters his details to login and continue using the app.
			\item The login information will be used to find games that needs scoring.
		\end{itemize}
	\subsection{Load Info}

	\subsection{Game Time}
	This use case is used to log the start and end time of each half of a match, as well as any intervals during which time was lost (the game was paused) and a reason for this time loss (injury, substitution, referee consultation, replacement of damaged player clothing). 
	\begin{itemize}Usage:
		\item The game time is displayed at the top of the UI.
		\item One starts the clock by tapping on it once. Preferably at kickoff. In the future the user will be presented with a confirm option in the form of a pop up which will not affect the running clock if accepted and reset the clock if denied.
		\item Once the clock is running a single tap on the clock will register a pause and activate a pop up asking the user for a reason for the pause, along with the legal options. The user will also have the option return should they accidentally activate this function. After a legal reason has been given for the pause the user is presented with a blank page and a button to resume the game.
		\item Upon reaching 40 minutes the game clock:
		\begin{itemize}
		\item registers the end of the current half,
		\item resets for the next half if there is one, and
		\item resets for overtime if conditions for it are met.
		\end{itemize}
	\end{itemize}

	\subsection{Scoring}
		During game time (i.e. while the OnlyRugby app is being used to log the information of a match currently being played ) the Scoring function will be used each time a player from either team scores points for their team (be it through a try, penalty kick, dropkick or a conversion kick following a try). The Scoring function works as follows:
		\begin{enumerate}
			\item Once the user has pressed the button to log score information they are redirected to a page which asks them to indicate which team scored by pressing either of two buttons which each have the names of the respective teams on them.
			\item The user is then redirected to a page which asks the user to specify what type of score it should log. This page displays buttons titled 'Try', 'Penalty Kick' and  'Dropkick' (note that 'Conversion Kick' is not an option as it can only be attempted after a succesful try). 
			\item The app then redirects the user to another page which has 15 buttons representing the 15 on-field players of the previously selected team. Each button has the player's position/jersey number on it as well as the player's name (if that information is provided on the database). This page also has an 'Unknown' button incase the user, for some reason, is not sure who scored on this team. The user is asked to select one of these players/buttons as the player who scored.
			\item In most cases the user is then done entering the score information but in the case of a 'Try' more steps must be completed:
				\begin{itemize}
					\item The user is first asked by a small pop-up message if the try was assisted by any other players with the options of clicking 'Yes' or 'No'. In the case of 'Yes' the user is then redirected to the same 'player selection page' as previously described were the user can then select a player which assisted the first player to score the try.
					\item The player is then asked by another small pop-up message if the subsequent conversion kick was successful with the options of clicking 'Yes' or 'No'. Selecting 'Yes' once again redirects the user to the 'player selection page' to select a player which successfully scored the conversion kick.
				\end{itemize}
		\end{enumerate}
	On all of the above described pages there are also buttons labeled 'Cancel' (to cancel logging the score) and 'Back' which redirects the user to the previous page and ignores the information that was logged on the page it is now leaving.

	\subsection{Substitutions}
		During game time the Substitutions function will be used to log any changes made to both the on- and off-field players (i.e. reserve players) of either team. This function will work as follows:
		\begin{enumerate}
			\item Once the player has pressed the button to log player substitutions they are redirected to a page which asks them to indicate which team is substituting a player by pressing either of two buttons which each have the names of the respective teams on them.
			\item The app then redirects the user to the 'player selection page' (as described in the Scoring section) which comprises of 15 buttons representing the on-field players of the selected team (i.e. without the 'Unknown' button). The user is then asked to select an on-field player which is being swapped out for a reserve player.
			\item The app then redirects the user to a page similar to the standard 'player selection page' which has between 0 and 8 buttons representing the reserve players of the selected team. These buttons also have the player's jersey numbers and names on them (if they are stored on the server's database). The user is then asked to select a reserve player to swap out with the previously selected on-field player. The two selected players then swap their places (i.e. the reserve is now on-field and vice versa). There is also an 'Unknown' button on this page which is used when a reserve is put on field which was not part of the reserve team which is stored on the database (which means the app therefore does not have information about him).
			\item When the player clicks the 'Unknown' button it displays a small pop-up menu which has two text-boxes and two buttons on it. The two text-boxes are there for the user to enter the new player's name and jersey number (either or both of these can be left empty). The two buttons are 'OK' which the user presses when they have completed the text-boxes to their satisfaction and 'Cancel' which cancels the process started by pressing the 'Unknown' button.
	On all of the above described pages there are also buttons labeled 'Cancel' (to cancel logging the substitution) and 'Back' which redirects the user to the previous page and ignores the information that was logged on the page it is now leaving.

	\subsection{Discipline}
		During game time (i.e. while the OnlyRugby app is being used to log the information of a match currently being played ) the Discipline function will be used each time a player performs an infraction that causes them to reveive a card. The Discipline function works as follows:
		\begin{enumerate}
			\item Once the user has pressed the button to log the disciplinary action's information they are redirected to a page which asks them to indicate which team's player was given a card, by pressing either of two buttons which each have the names of the respective teams on them.
			\item The app then redirects the user to another page which has 15 buttons representing the 15 on-field players of the previously selected team. Each button has the player's position/jersey number on it as well as the player's name (if that information is provided on the database). This page also has an 'Unknown' button incase the user, for some reason, is not sure who the player being disciplined is. The user is asked to select one of these players/buttons as the player who was disciplined.
			\item The app then redirects the user to another page which has 3 buttons, to indicate if a white, yellow or red card was given.
		\end{enumerate}
	On all of the above described pages there are also buttons labeled 'Cancel' (to cancel logging the discipline) and 'Back' which redirects the user to the previous page and ignores the information that was logged on the page it is now leaving.

	\subsection{Line-outs}
	Currently not implemented. Usage likely to be similar to the following description:
	\begin{itemize}
		\item Line out button present on landing page after login.
		\item Button can only be activated during game time.
		\item Activating the button registers a line out and redirects the interface to the line outs page. Here the user will be presented with options to specify:
			\begin{enumerate}
				\item The team to which the line out was given.
				\item The player to perform the throw in, by player number or by name. (Defaults to "unknown player").
				\item The team that won the line out.
			\end{enumerate}
		\item An option to return to the landing page will also be present
		\item Activating the return button will present a pop up prompting the user to save data as default, what the user entered, or to cancel the line out.
	\end{itemize}

	\subsection{Scrums}
		\begin{itemize}
			\item The scrum button will be on the game page when a game scoring is in progress.
			\item It will then ask the reason for the scrum and which team the scrum belongs to.
			\item It will then ask who won the scrum.
		\end{itemize}
		
	\subsection{Tackles}

	\subsection{Possession}

	\subsection{Turnovers}
		During game time (i.e. while the OnlyRugby app is being used to log the information of a match currently being played ) the Turnover function will be used each time a player from either team manages to take possession of the ball from an opposing player.The Turnover function works as follows:
		\begin{enumerate}
			\item Once the user has pressed the button to log the turnover's information they are redirected to a page which asks them to indicate which team won the turnover by pressing either of two buttons which each
			have the names of the respective teams on them.
			\item The app then redirects the user to another page which has 15 buttons representing the 15 on-field players of the previously selected team. Each button has the player's position/jersey number on it 
			as well as the player's name (if that information is provided on the database). This page also has an 'Unknown' button incase the user, for some reason, is not sure who won the turnover. The user is asked to select one of these players/buttons as the player who won the turnover.
			\item The user is then redirected to a page, similar to the one above, that has a list of the other team's 15 on-field players. Here, the user selects the player who lost the ruck.
		\end{enumerate}
	On all of the above described pages there are also buttons labeled 'Cancel' (to cancel logging the turnover) and 'Back' which redirects the user to the previous page and ignores the information that was logged on the page it is now leaving.

	\subsection{Clean break}
		\begin{itemize}
			\item The clean break button will be on the game page when a game scoring is in progress.
			\item It will then ask the player that performed the clean break.
		\end{itemize}

	\subsection{Offloads}
	Currently not implemented. Usage likely to be similar to the following description:
	\begin{itemize}
		\item Offload button present on landing page after login.
		\item Button can only be activated during game time.
		\item Activating the button registers an offload and presents a pop up prompt asking for:
			\begin{enumerate}
				\item The team to which the offload was given. (2 buttons)
				\item The player that performed the offload, by player number or by name. (Defaults to "unknown player").
			\end{enumerate}
		\item An option to cancel the offload will also be present along with return option
		\item Activating the return button will prompt the user to save data as default, what the user entered, or to cancel the registering of an offload.

	\subsection{Ruck}
		During game time (i.e. while the OnlyRugby app is being used to log the information of a match currently being played ) the Ruck function will be used each time the ball is on the ground and players are crowded around it.During this time, they must use their feet to maneuver it to the back of the rear player's feet, at which point it may be picked up. The Ruck function works as follows:
		\begin{enumerate}
			\item Once the user has pressed the button to log the ruck's information they are redirected to a page which asks them to indicate which
			team won the ruck by pressing either of two buttons which each have the names of the respective teams on them.
		\end{enumerate}
	On the above described page there are also buttons labeled 'Cancel' (to cancel logging the ruck) and 'Back' which redirects the user to the previous page and ignores the information that was logged on the page it is now leaving.

	\subsection{Maul}

\subsection{Troubleshooting}
		
\end{document}
