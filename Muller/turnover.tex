\subsection{Turnover}
		During game time (i.e. while the OnlyRugby app is being used to log the information of a match currently being played ) 
		the Turnover function will be used each time a player from either team manages to take possession of the ball from an opposing player.
		The Turnover function works as follows:
		\begin{enumerate}
			\item Once the user has pressed the button to log the turnover's information they are redirected to a page which asks them to indicate which team won the turnover by pressing either of two buttons which each
			have the names of the respective teams on them.
			\item The app then redirects the user to another page which has 15 buttons representing the 15 on-field players of the previously selected team. Each button has the player's position/jersey number on it 
			as well as the player's name (if that information is provided on the database). This page also has an 'Unknown' button incase the user, for some reason, is not sure who won the turnover. The user is asked to select one of these players/buttons as the player who won the turnover.
			\item The user is then redirected to a page, similar to the one above, that has a list of the other team's 15 on-field players. Here, the user selects the player who lost the ruck.
		\end{enumerate}
	On all of the above described pages there are also buttons labeled 'Cancel' (to cancel logging the turnover) and 'Back' which redirects the user to the previous page and ignores the information that was logged on the page it is now leaving.